\documentclass[12pt,a4paper,titlepage]{article}
\usepackage[czech]{babel}
\usepackage[utf8]{inputenc}
\renewcommand{\refname}{Literatura}
\title{Maturintí práce z informatiky:\\
	Aplikace pro syntézu hudby z formátu MusicXML do formátu SMF (MIDI) pomocí GNU Octave}
\author{Alena Smutná\\
	R8.A\\
	Gymnázium Jana Keplera\\}
\begin{document}
\maketitle
\tableofcontents
\newpage
\section{Úvod}
Tato práce se zabývá zpracováním hudebních skladeb ve formátu MusicXML a jejich přeměnou do zvukové podoby ve formátu MIDI. Program je psán v jazyce GNU Octave, s využitím funkcí knihovny Xerces2 Java Parser 2.12.0 pro načítání a parsování XML dokumentu a funkce xml2struct z Mathworks File Exchange pro jeho konverzi z XML Document Object Model do datového typu struktura.
\newpage
\section{Teoretická část}
\subsection{XML}
XML, nezkráceně \uv{eXtensible Markup Language} je značkovací jazyk, specifikovaný společností W3, který je strojově zpracovatelný, a zároveň je čitelný pro lidi. Užívá se především pro přenos dat mezi různými aplikacemi a pro uchovávání dat, u kterých je důležitá struktura a obsah jednotlivých částí. XML dokumenty jsou textové dokumenty užívající kódování Unicode.
\\
Specifikace XML formátu nedefinuje jednotlivé značky, ty jsou definovány buď v jednotlivých konkrétních aplikacích (např. XHTML, RSS, SVG nebo MusicXML), nebo se dá definovat vlastní sada značek pomocí tzv. definičních jazyků (např. DTD). 
\subsubsection{MusicXML}
MusicXML je formát založený na XML pro zápis hudebních skladeb v západní notaci. Byl vytvořen pro snadný přenos skladeb mezi jednotlivými aplikacemi pro zápis skladeb (nejčastěji vizuálními). MusicXML soubory obsahují informace o vizuální podobě skladby i informace o použitých nástrojích, které se využívají při konverzi skladeb do zvukových formátů.
\subsection{MIDI}
MIDI, nezkráceně \uv{Musical Instrument Digital Interface} je specifikace pro propojování různých hudebních zařízení. Specifikuje jednak hardwarové propojení zařízení, jednak komunikační a datový protokol. Komunikace probíhá pomocí tzv. zpráv, které popisují jednotlivé události (např. začátek či konec noty).
\subsubsection{General MIDI}
Specifikace General MIDI rozšiřuje MIDI specifikaci přenosových protokolů o polyfonii, zavádí standardizované zvukové programy (zvuky jednotlivých nástrojů) a přidává některé kontrolní zprávy.
\subsubsection{SMF}
SMF, nezkrceně \uv{Standard MIDI File} je formát souborů, zapsaných dle General MIDI specifikace, sloužících pro uložení hudebních dat ve formě událostí a nikoliv jednotlivých zvuků, jako je tomu u běžných formátů používaných pro uložení zvuků. Je určen pro přenos dat mezi jednotlivými zařízeními. SMF soubor se skládá z jednotlivých stop, které je možné přehrát současně a které obsahují zprávy, které popisujují jednotlivé události. Do SMF souborů se dají ukládat i informace o skladbě či autorovi, pomocí tzv. meta-událostí. Každá událost se skládá z časového přírůstku (od poslední události), stavového bytu, který udává o jakou událost půjde, a datového bytu, který obsahuje parametry události.
\subsection{GNU Octave}
GNU Octave je matematicky orientovaný vyšší programovací jazyk, značně kompatibilní s jazykem Matlab, ale na rozdíl od něj je svobodný. K ukládání dat se nejčastěji používají matice a GNU OCtave podporuje velké množství maticových operací. GNU Octave podporuje i ukládání dat do taz. datových struktur, pomocí kterých se dají vytvářet jakési \uv{stromy} dat v paměti.
\section{Návod k použití}
\section{Implementace}
implementacni zajimavosti - je treba zprovoznit a nakonfigurovat knihovny
indexovani a prochazeni struktury
ne vzdy je u noty id nastroje, kdyz je v part jen jeden, je vynechan, kdyz ma hlas pauzu, je id nastroje taky vynechano
\section{Testování}
nefunguje hlasitost, zmeny tempa, pravdepodobne ozdoby, ani trioly
\section{Závěr}
\addcontentsline{toc}{toc}{Literatura}
\begin{thebibliography}{10}
\bibitem{xml-en}en.wikipedia.org/wiki/XML
\bibitem{xml-cs}cs.wikipedia.org/wiki/Extensible\_Markup\_Language
\bibitem{mxml}musicxml.com
\bibitem{mxml-usermanual}usermanuals.musicxml.com/MusicXML/Content/Contents.htm
\bibitem{midi}root.cz/clanky/rozhrani-midi-na-osobnich-pocitacich/
\bibitem{gm}root.cz/clanky/general-midi-a-format-souboru-smf/\#k03
\bibitem{smf-cs}cs.wikipedia.org/wiki/SMF
\bibitem{smf}csie.ntu.edu.tw/~r92092/ref/midi/
\end{thebibliography}
\end{document}
